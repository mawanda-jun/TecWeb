\section{I nostri campi}
L'azienda agricola copre una superficie di ((quanti km quadrati?)) e vanta la coltivazione di sementi antiche di grano con tecniche all'avanguardia, senza l'utilizzo di pesticidi e fertilizzanti chimici.

\subsection{Il grano: vecchio e nuovo}
Durante gli anni '70 il grano duro ha subito una spinta genetica notevole, che da un lato lo ha reso più basso - quindi più difficilmente abbattibile e più facilmente raccoglibile dai macchinari moderni -, dall'altro lo ha reso più produttivo per la produzione di farine e derivati.
La nostra azienda ha tuttavia deciso di seguire una strada diversa: ricerche importanti hanno visto come il grano duro antico offra notevoli vantaggi nutrizionali. Infatti il glutine in esso contenuto ha una struttura diversa dagli altri e più facilmente digeribile, oltre ad avere una composizione proteica migliore ed un impatto glicemico inferiore.
Non sono poi da trascurare i loro importanti contributi alla biodiversità e all'ambiente. Va infatti considerato che i grani di antiche varietà hanno meno bisogno di concimazioni, altrimenti si sviluppano anche troppo in altezza; essendo belli svettanti, sono più difficilmente attaccabili dalle infestanti, rendendo superflui i diserbanti. Insomma, si prestano ottimamente alla coltivazione biologica.((ci si può dilungare))
Ecco e nostre sementi:
\begin{itemize}
	\item \textbf{Timilia o grano marzuolo}: Grano duro siciliano già citato in epoca greca. È particolarmente apprezzato per la panificazione grazie al gusto dolce e al colore carico della farina. Se ne ricava tra l'altro il celebre pane Nero di Castelvetrano. ((img: grano-timilia))
	\item \textbf{Frassineto}: Nato nel 1922 nell'Aretino, è un grano tenero derivato dal Gentil Rosso (vedi). Vanta gusto e aroma intensi. ((img: grano-frassineto))
	\item \textbf{Senatore Cappelli}: Nato con l'intento di aumentare la produzione di frumento per la crescente popolazione, deve il nome al senatore abruzzese Raffaele Cappelli, nella cui tenuta il genetista Nazareno Strampelli compì gli incroci che lo portarono, dopo un'ibridazione con una cultivar tunisina, a produrre un grano di maggiore resa. Non per questo il Cappelli perse sotto il profilo nutrizionale e della digeribilità. ((img: grano-senatore-cappelli))
	\item \textbf{Dicocco o farro medio}: Proteico e ricco di antiossidanti, appartiene davvero a un'antica cultivar ed era il preferito dagli antichi romani. ((img: grano-triticum-dicoccum)) 
	\item \textbf{Gentil Rosso}: Nato in Toscana a metà '800, questo grano tenero ha spighe rossicce da cui si ricava una farina di colore carico. È ben fornito di minerali e proteine. ((img: grano-gentil-rosso))
	\item \textbf{Rieti}: Originario dell'omonima città laziale, era già coltivato nel '600, ma ebbe grande diffusione in Italia nell’800. ((img: grano-rieti))
	\item \textbf{Solina}: Grano tenero diffuso in Abruzzo già nel XVI secolo. ((img: grano-solina))
	\item \textbf{Russello}: Pregiata varietà di grano duro siciliano, deve il suo nome al colore rosso-dorato delle spighe. Ben digeribile, è molto apprezzato per la panificazione. ((img: grano-russello))
	\item \textbf{Verna}: Grano tenero originario della Toscana. Sottoposta ad analisi dall'università di Bologna, la farina ha mostrato un tenore di glutine dello 0,9\% contro il 14\% della media delle attuali cultivar, rispetto alle quali ha pure un miglior contenuto di antiossidanti, proteine totali, minerali.
\end{itemize}