\newpage
\section{Validazione}
Al fine di creare pagine il più possibile accessibili da parte di più tipologie di utente si sono potuti sfruttare alcuni strumenti automatici di validazione e si sono prese alcune scelte progetturali.
\subsection{Scelte progetturali}
Si è deciso di rendere le pagine di utenza generica il più possibile accessibili, ad esempio eliminando JavaScript dall'utilizzo desktop e rendendolo marginale in quello mobile. Invece si è optato per una produzione meno accorta per le pagine di amministrazione. Questo perché si suppone che l'amministratore - in quanto dipendente dell'azienda - sia in un ambiente controllato e con tutte le tecnologie richieste da sito attive.
\\Per rendere le pagine più accessibili è stata presa la decisione di associare ad alcuni punti nevralgici della pagina dei \emph{tabindex}. Questa può essere considerata una scelta rischiosa e ci è stata segnalata dagli strumenti automatici che poi abbiamo utilizzato, tuttavia abbiamo deciso di sfruttarla comunque principalmente per la possibilità, da parte di un utente non vedente, di poter "saltare" direttamente al contenuto o ad altre ancore o link presenti nel sito, secondo un ordine prestabilito.
\subsection{Strumenti automatici}
\subsubsection{W3C Validator plugin for Visual Studio Code}
Questo plugin è stato fondamentale durante la stesura del documento, in quanto in tempo reale ci si è resi conto dei problemi della scrittura del codice HTML.
\subsubsection{Markup Validation Service w3.org}
Tutte le pagine dell'utente generico sono state validate attraverso questo validatore. Rimane inrisolto un errore riguardante il tag \emph{target} presente nei riferimenti, invece richiesto dal validatore \emph{Vamola} di cui parleremo più tardi. Inoltre c'è da segnalare un warning ”Using Direct Input mode: UTF-8 character encoding assumed” per l’utilizzo della modalità di input diretto del codice HTML.
\subsubsection{W3C CSS Validator w3.org}
Tutti i fogli di stile sono stati validati con questo validatore e non hanno riportato errori o warnings.
\subsubsection{Vamolà}
Questo strumento è stato utilizzato per la validazione secondo le linee Guida: WCAG 2.0 (Level AAA). Le pagine non hanno riportato errori. Come riportato prima si è preferito aggiungere il tag \emph{target} per risolvere i problemi di accessibilità qui segnalati.
\subsubsection{TotalValidator}
Tutte le pagine sono state validate utilizzando Total Validator, che ha riportato solo gli errori relativi a \emph{target} di cui prima.
\subsubsection{ColorSafe}
Per la generazione di una palette di colori accessibile alla maggior parte dei disturbi visivi è stato scelto di utilizzare questo sito (\url(http://colorsafe.co/)) che, a partire da un colore di background, rende disponibili alcuni colori secondo gli standard WCAG 2.0 Level AAA.
\subsection{Test effettuati}
Il sito è stato testato con successo su Chrome, Firefox, Edge e Internet Explorer versione 6.
\\Inoltre sono state effettuate delle dei test anche per la versione mobile, in particolare su:
\begin{itemize}
	\item Huawei P9 Lite;
	\item Xiaomi Mi Mix;
	\item Asus Zenfone 2;
\end{itemize}
su molteplici browser mobile. L'usabilità della parte di utenza generica rimane invariata, in quanto responsive. Invece, come progettato, l'usabilità della parte amministrativa è limitata in quanto non è completamente responsive. Non è infatti pensata per un target di utilizzatori da mobile ma per la gestione di un'attività di ufficio. Da notare tuttavia che, da tablet, l'applicativo è utilizzabile.